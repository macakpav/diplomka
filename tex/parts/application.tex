
%!TEX ROOT=../_main.tex

\chapter{Tvarova optimalizace kompresorove mrize}

chceme optimalizovat stlaceni

\section{Sdruzene rce pro cilovou fci stlaceni}
konkretni postup odvozeni rovnic z "nove" cilove funkce, zjednoduseni cilove fce pro nestlacitelne proudeni (pouze rozdil bez podilu)

Stlaceni
\begin{equation}\label{key}
	\Pi = \dfrac{p_{IN}^{t}}{p_{OUT}^{t}}
\end{equation}
Minimalizovanou cílovou funkci lze tedy zapsat jako
\begin{equation}\label{key}
	J = p_{OUT}^{t}- p_{IN}^{t},
\end{equation}
kde průměrný celkový tlak na libovolné hranici $ \Gamma $ získáme pomocí integrálu
\begin{equation}\label{key}
	p^t_\Gamma = \int_\Gamma p+\dfrac{1}{2}(\mathbf{u}\cdot \mathbf{u}) \, \mathrm{d}S.
\end{equation}
Ve smyslu vztahu \ref{eq:cenova_fce} má takto definovaná cílová funkce stlačení pouze hraniční složku $ J_\Gamma $ a pro sdružené rovnice tak bude figurovat pouze v hraničních členech, tedy v rovnicích \ref{eq:sdruzenaOP1} a \ref{eq:sdruzenaOP2}. Potřebujeme tedy vydefinovat parciální derivace podle primárních proměnných $ \mathbf{u} $ a $ p $.

Parciální derivaci podle skalární veličiny $p$ lze psát přímo jako
\begin{equation}\label{key}
\dfrac{\partial J_\Gamma}{\partial p}
=
\dfrac
{\partial 
	\left( 
		p + \frac{1}{2} \mathbf{u}\cdot\mathbf{u} 
	\right)}
{\partial p}
= 1.
\end{equation}

Derivaci podle rychlosti můžeme psát jako
\begin{equation}\label{key}
\dfrac{\partial J_\Gamma}{\partial \mathbf{u}}
=
\dfrac{\partial \left( p + \frac{1}{2} \mathbf{u}\cdot\mathbf{u} \right)}{\partial \mathbf{u}}
=
\frac{1}{2}\dfrac{\partial \left( u_j u_j \right)}{\partial u_i} = \dfrac{\partial  u_j  }{\partial u_i}u_j = \delta_{ij}u_j=u_i=\mathbf{u}.
\end{equation}
Pro pozdější implementaci v softwarovém balíku OpenFOAM je navíc potřeba vydefinovat ještě následující derivace.
\begin{equation}\label{key}
\frac{\partial J_\Gamma}{\partial u_n} = 
\dfrac{\partial \left( p + \frac{1}{2} \mathbf{u}\cdot\mathbf{u} \right)}{\partial u}
\end{equation}

vydefinovani clenu potrebnych pro OF

implementace v OF 


\section{Optimalizace mrize GHH 1-S1}

popis geometrie, vypocetni oblasti, okrajove podminky, nastaveni optimalizacniho algoritmu

Pro aplikaci optimalizačního algoritmu s novou cenovou funkcí pro stlačení byla zvolena axiální kompresorová mříž MAN GHH 1-S1 publikovaná v \cite{steinert1990design}. Výpočetní oblast 

\subsection{vysledky}
klaibrace S-A modelu, prubeh cilove fce, overeni vysledku pomoci vhodnejsiho modelu turbulence
