% arara: pdflatex: { synctex: yes }
% arara: makeindex -s ctuthesis.ist -t _main.nlg -o _main.nls _main.nlo
% arara: bibtex
% arara: nomencl
% arara: pdflatex
% makeindex -s ctuthesis.ist _main

% The class takes all the key=value arguments that \ctusetup does,
% and a couple more: draft and oneside
\documentclass[twoside]{ctuthesis}

\ctusetup{
	preprint = \ctuverlog,
%	mainlanguage = english,
	titlelanguage = czech,
	mainlanguage = czech,
	otherlanguages = {slovak,english},
	title-czech = {Tvarová optimalizace lopatkové mříže sdruženou metodou},
	title-english = {Shape optimization of blade cascade with adjoint method},
	subtitle-czech = {},
	subtitle-english = {},
	doctype = M,
	faculty = F2,
	department-czech = {Ústav technické matematiky},
	department-english = {Department of Mathematics},
	author = {Bc. Pavel Mačák},
	supervisor = {doc. Ing. Jiří Fürst, Ph.D.},
	supervisor-address = { Ústav technické matematiky \\ Resslova 307/9 \\ Praha 6},
	%supervisor-specialist = {John Doe},
	fieldofstudy-english = {Mathematical modelling},
	subfieldofstudy-english = {Applied Sciences in Mechanical Engineering},
	fieldofstudy-czech = {Matematické modelování v technice},
	subfieldofstudy-czech = {Aplikované vědy ve strojním inženýrství},
	keywords-czech = {Optimalizace, CFD, OpenFOAM, Lopatková mříž},
	keywords-english = {Optimization, CFD, OpenFOAM, Compressor cascade},
	day = 1,
	month = 1,
	year = 2022,
	specification-file = {others/zadani_BP.pdf},
	front-specification = true,
%	front-list-of-figures = false,
%	front-list-of-tables = false,
%	monochrome = true,
	layout-short = true,
}

\ctuprocess

\addto\ctucaptionsczech{%
	\def\supervisorname{Vedoucí}%
	\def\subfieldofstudyname{Studijní program}%
}

\ctutemplateset{maketitle twocolumn default}{
	\begin{twocolumnfrontmatterpage}
		\ctutemplate{twocolumn.thanks}
		\ctutemplate{twocolumn.declaration}
		\ctutemplate{twocolumn.abstract.in.titlelanguage}
		\ctutemplate{twocolumn.abstract.in.secondlanguage}
		\ctutemplate{twocolumn.tableofcontents}
		\ctutemplate{twocolumn.listoffigures}
	\end{twocolumnfrontmatterpage}
}

% Theorem declarations, this is the reasonable default, anybody can do what they wish.
% If you prefer theorems in italics rather than slanted, use \theoremstyle{plainit}
\theoremstyle{plainit} 
\newtheorem{theorem}{Theorem}[chapter]
\newtheorem{corollary}[theorem]{Corollary}
\newtheorem{lemma}[theorem]{Lemma}
\newtheorem{proposition}[theorem]{Proposition}
\newtheorem{definition}[theorem]{Definice}
\newtheorem{problem}[theorem]{Problém}

\theoremstyle{definition}
\newtheorem{example}[theorem]{Example}
\newtheorem{conjecture}[theorem]{Conjecture}

\theoremstyle{note}
\newtheorem*{remark*}{Remark}
\newtheorem{remark}[theorem]{Remark}

\setlength{\parskip}{5ex plus 0.2ex minus 0.2ex}

\clubpenalty 10000  %prvni radek odstavce nebude sam na konci stranky (vdova) 
\widowpenalty 10000 %posledni radek odstavce nepujde na novou stranku  (sirotek) 


% Abstract in Czech
\begin{abstract-czech}
	\lipsum[1]

\end{abstract-czech}

% Abstract in English
\begin{abstract-english}
	\lipsum[1]

\end{abstract-english}

% Acknowledgements / Podekovani
\begin{thanks}
Chtěl bych poděkovat svému vedoucímu práce doc. Ing. Jiřímu Fürstovi, Ph.D. za odborné vedení, za pomoc a rady při zpracování této práce. Zároveň děkuji své rodině a přátelům za jejich podporu při studiu.
\end{thanks}

% Declaration / Prohlaseni
\begin{declaration}
Prohlašuji, že jsem předloženou práci vypracoval samostatně a uvedl veškerou použitou literaturu.

V Praze, \ctufield{day}.~\monthinlanguage{title}~\ctufield{year}
\end{declaration}

% Only for testing purposes
\listfiles
\usepackage{amsmath, gensymb}
\usepackage[pagewise]{lineno}
\usepackage{lipsum,blindtext}
\usepackage{mathrsfs} % provides \mathscr used in the ridiculous examples
\usepackage{mathtools}

\usepackage{scrextend}

\usepackage[]{nomencl} 
\makenomenclature
\renewcommand{\nomname}{Seznam použitých symbolů a zkratek}
% \nomenclature[⟨prefix⟩]{⟨symbol⟩}{⟨description⟩}


%\usepackage{imakeidx}
\makeindex
		% makeindex -s ctuthesis.ist _main
\begin{document}


\maketitle


\chapter{Úvod}

%!TEX ROOT=../_main.tex

This is uvod

\lipsum[1-3] \cite{cite:bible}

\part{Teoretická část}
%!TEX ROOT=../_main.tex

\chapter{Základy numerického řešení nestlačitelných NS rovnic}
%!TEX ROOT=../../_main.tex

Problém proudění či mechanika tekutiny je v rámci této práce chápán jako zkoumání pohybu velkého množství částic a jejich interakce. Velké množství ve smyslu, že zkoumané fluidum má takovou hustotu, že lze použít aproximaci reality pomocí matematického kontinua. To nám říká, že i v nekonečně malá (infinitesimální) část tekutiny obsahuje dostatečný počet částic, pro které lze specifikovat střední rychlost a střední kinetickou energii. Jsme tak schopni definovat pojmy rychlost, tlak, teplota, hustota a další důležité veličiny jako spojité funkce v rámci celého kontinua. Tato kapitola vychází různou měrou z publikací \cite{blazek2015computational, dvorak1987vnitrniaerodynamika, hirsch2007numerical, shapiro1953dynamics, furst2020mko2}

\section{Základy matematického popisu proudění} \label{sec:zaklady_popisu}

Odvození základních rovnic mechaniky tekutin se opírá tzv. zákony zachování. Pro případ obecné tekutiny to jsou
\begin{enumerate}
	\item zachování hmoty
	\item zachování hybnosti a
	\item zachování energie.
\end{enumerate}
Pro případ nestlačitelné tekutiny si pak vystačíme pouze s prvními dvěma zmíněnými zákony zachování.

Zachování určité veličiny znamená, že její časovou změnu uvnitř libovolného objemu lze vyjádřit jako množství veličiny proudící přes hranici zvoleného objemu a produkci veličiny uvnitř objemu. Často se také mluví o bilanci veličiny v určitém objemu. Množství veličiny, které proudí přes hranici objemu se nazývá tok. Obecně se tok dá rozdělit na dvě složky. Konvekci, způsobenou konvektivním přenosem veličiny, a difuzi, způsobenou pohybem molekul tekutiny v klidovém stavu. Difuzivní přenos závisí na gradientu dané veličiny a pro případ homogenní distribuce tedy vymizí.

\subsection{Kontrolní objem a zákon zachování}\label{sec:kontrolni_objem}
V předchozí kapitole se o zákonech zachování mluvilo v kontextu jistého objemu. Takovémuto libovolně zvolitelnému objemu se často říká kontrolní objem, nebo - pro účely numerické matematiky vhodněji - konečný kontrolní objem.

Mějme obecný kontrolní objem $\Omega$ s uzavřenou hranicí $\Gamma$, který je pevný v prostoru s daným proudovým polem jak naznačuje obrázek \ref{fig:kontrolni-objem}. Zároveň lze definovat element hranice $\mathrm{d}S$ a jeho vnější normálu $\mathbf{n}$.
\begin{figure}[h]
	\def\svgwidth{0.8\textwidth}
	\def\svgwidth{\columnwidth}
	\graphicspath{{img/inkscape/}}
	\includesvg{img/inkscape/kontr_objem}
	%\includegraphics[width=0.7\textwidth]{img/inkscape/drawing.eps}
	\caption{Pevný kontrolní objem v obecném proudovém poli.}
	\label{fig:kontrolni-objem}
\end{figure}

Pro obecnou zachovávanou veličinu $W$ lze zákon zachování psát jako
\begin{equation}\label{eq:zachovani}
\dfrac{\partial}{\partial t} \int_{\Omega}W\,\mathrm{d}V + \int_{\Gamma}\mathbf{F}(W) \cdot \mathbf{n} \, \mathrm{d}S = \int_{\Omega}Q_\Omega(W) \, \mathrm{d}V + \int_{\Gamma} \mathbf{Q_\Gamma}(W) \cdot \mathbf{n} \, \mathrm{d}S,
\end{equation}
kde $Q_{\Omega}(W)$ jsou objemové a  $\mathbf{Q_\Gamma}(W)$ povrchové zdroje a $\mathbf{F}(W) $ je vektor hustoty toku veličiny $W$ plochou $\Gamma$. Zákon v této formě je formálně platný jak pro skalární veličinu $W$ tak vektorovou $\mathbf{W}$. Speciálně pak pro skalární veličinu lze člen s tokem přes hranici rozdělit, podle dříve zmíněného dělení, na konvektivní tok
\begin{equation}\label{eq:konv_tok}
\mathbf{F_K}(W) = W\mathbf{u}
\end{equation}
a difuzivní tok vyjádřený pomocí zobecněného Fickova gradientního zákona
\begin{equation}\label{eq:diff_tok}
\mathbf{F_D}(W) = \kappa \rho \nabla(W/\rho),
\end{equation}
kde $\kappa$ je koeficient difuzivity a dohromady tedy
\begin{equation}
\int_{\Gamma}\mathbf{F}(W) \cdot \mathbf{n} \, \mathrm{d}S = \int_{\Gamma}W\left[\mathbf{u}\cdot\mathbf{n}\right] - \kappa \rho \left[\nabla(W/\rho)\cdot \mathbf{n} \right] \mathrm{d}S.
\end{equation}
Rovnici \ref{eq:zachovani} tak můžeme rozepsat do podoby
\begin{equation}\label{eq:zachovani_skalarStoky}
\dfrac{\partial}{\partial t} \int_{\Omega}W\,\mathrm{d}V + \int_{\Gamma}W\left[\mathbf{u}\cdot\mathbf{n}\right] - \kappa \rho \left[\nabla(W/\rho)\cdot \mathbf{n} \right] \mathrm{d}S = \int_{\Omega}Q_\Omega(W) \, \mathrm{d}V + \int_{\Gamma} \mathbf{Q_\Gamma}(W) \cdot \mathbf{n} \, \mathrm{d}S.
\end{equation}
Pro vektorovou veličinu lze udělat velmi podobné rozdělení, pouze s tím rozdílem, že všechny tři funkce $W$ ($\mathbf{F}, Q_{\Omega}, \mathbf{Q_\Gamma}$) budou o jeden tenzorový řád vyšší. Rovnice \ref{eq:zachovani} s rozdělením na tenzory konvektivního a difuzivního toku tak dostane podobu
\begin{equation}\label{eq:zachovani_vektor}
\dfrac{\partial}{\partial t} \int_{\Omega}\mathbf{W} \, \mathrm{d}V + \int_{\Gamma}\left(\mathbb{F}_K(\mathbf{W})-\mathbb{F}_D(\mathbf{W}) \right)\cdot \mathbf{n} \, \mathrm{d}S = \int_{\Omega} \mathbf{Q_\Omega}(\mathbf{W}) \, \mathrm{d}V + \int_{\Gamma} \mathbb{Q}_\Gamma(\mathbf{W}) \cdot \mathbf{n} \, \mathrm{d}S.
\end{equation}
Takto odvozený obecný zákon zachování (někdy taky bilanční rovnici) lze využít pro odvození základních rovnic proudění.



\subsection{Zákon zachování hmoty, Rovnice kontinuity}
Pro jednosložkové tekutiny vyjadřuje zákon zachování hmoty, tedy že hmotu v systému nelze vytvořit, ani ztratit, i.e. zdroj hmoty se uvnitř kontrolního nepředpokládá. Musí tedy platit, že změna hmotnosti uvnitř kontrolního objemu musí být rovna toku hmoty přes hranice kontrolního objemu, tedy
\begin{equation}
-\dfrac{\partial}{\partial t}\int_\Omega \rho \,\mathrm{d}V = \int_\Gamma \rho \, u_i \, n_i \mathrm{d}S = \int_\Omega \dfrac{\partial\left(\rho u_i\right)}{\partial x_i}\mathrm{d}V.
\end{equation}
\nomenclature[P]{$\dfrac{\partial}{\partial w}$}{Parciální derivace podle veličiny $w$}
\nomenclature[D]{$t$}{Čas}
\nomenclature[U]{$u_i,\, \mathbf{u}$}{Vektor rychlosti}
\nomenclature[N]{$n_i,\, \mathbf{n}$}{Normálový vektor hranice}
\nomenclature[R]{$\rho$}{Hustota}
\nomenclature[O]{$\Omega$}{Kontrolní objem}
\nomenclature[G]{$\Gamma$}{Hranice kontrolního objemu}
Po převedení obou integrálů na jednu stranu, záměně operací integrace a derivace a vyžití distributivity integrálu vzhledem k operaci součet, dostáváme obecný tvar rovnice kontinuity pro nestacionární proudění stlačitelné tekutiny
\begin{equation}\label{eq:kontinuita_stlacitelna}
\dfrac{\partial \rho}{\partial t} + \dfrac{\partial \left(\rho u_i\right)}{\partial x_i} = 0.
\end{equation}
Ke stejné rovnici dojdeme, pokud do rovnice zachovaní \ref{eq:zachovani_skalarStoky} dosadíme za obecnou skalární veličinu $W$ hustotu $\rho$, uplatníme předpoklad nulových zdrojů na pravé straně a uvědomíme si, že difuzivní tok z rovnice \ref{eq:diff_tok} bude nulový, neboť 
\begin{equation}
\nabla(W/\rho) = \nabla(\rho/\rho) = \nabla(1) = 0.
\end{equation} 
Za předpokladu nestlačitelnosti tekutiny, tedy že $\rho = konst.$ lze navíc rovnici \ref{eq:kontinuita_stlacitelna} zjednodušit na 
\begin{equation}\label{eq:kontinuita_nestlacitelna}
\dfrac{\partial u_i}{\partial x_i} = \nabla\cdot\mathbf{u} = 0,
\end{equation}
což se běžně označuje jako rovnice kontinuity pro proudění nestlačitelné tekutiny (v indexovém a vektorovém zápisu).

\subsection{Zákon zachování hybnosti, Rovnice hybnosti}
Odvození rovnice hybnosti vychází z druhého Newtonova zákona, který říká, že změna hybnosti je způsobena součtem sil účinkujících na element hmotnosti.
Hybnost nekonečně malé části kontrolního objemu je\begin{equation}
\rho \mathbf{u}\,\mathrm{d}V
\end{equation}
a tedy změna hybnosti uvnitř kontrolního objemu je
\begin{equation}
\dfrac{\partial}{\partial t}\int_\Omega\rho\mathbf{u}\,\mathrm{d}V.
\end{equation}
Sledovanou zachovávanou veličinou vektorovou veličinou $\mathbf{W}$ z analogie předchozího vztahu s prvním členem rovnice \ref{eq:zachovani_vektor} je hybnost $\rho \mathbf{u}$. Formálním použitím rovnice \ref{eq:konv_tok} dostáváme vztah pro tenzor konvektivního toku
\begin{equation}
\mathbb{F}_K(\mathbf{\rho\mathbf{u}})\cdot \mathbf{n}=\rho\mathbf{u} (\mathbf{u}\cdot\mathbf{n})
\end{equation}
Difuzivní tok zůstává nulový neboť hybnost nemůže difundovat v tekutině za klidového stavu.

Nejdůležitější částí odvození rovnice hybnosti je interpretace zdrojových členů. Zdroj hybnosti je z hlediska fyziky vždy síla.
\begin{enumerate}
	\item Objemové síly působí na hmotu v celém kontrolním objemu e.g. síla gravitační, inerciální, Coriolisova či elektromagnetická etc. 
	\item Povrchové síly působí přímo na povrchu $\Gamma$ kontrolního objemu. Jedná se o deformační působení vnějších sil. Tenzor napětí, kterým se často toto působení vyjadřuje lze rozdělit na sférickou a deviátorovou složku, které v případě tekutin lze interpretovat jako působení tlaku okolí a smykové a normálové napětí vznikající mezi okolím a kontrolním objemem.
\end{enumerate}

Objemové zdroje lze vyjádřit jednoduše. Pokud příslušnou vnější sílu vztáhneme na jednotku objemu $\rho \mathbf{f_e}$ lze psát
\begin{equation}
\int_{\Omega} \mathbf{Q_\Omega} \mathrm{d}V = \int_{\Omega} \rho \mathbf{f_e} \,\mathrm{d}V.
\end{equation}
Povrchové zdroje jsou rozdělené na sférické působení okolního tlaku $p$ a tenzor viskózního napětí $\tau$, tedy
\begin{align*}
\mathbb{Q}_\Gamma &= -p \mathbb{I}+\tau, \\
Q_{\Gamma ij}&= -p \delta_{ij}+\tau_{ij},
\end{align*}
kde $\mathbb{I}$ je jednotkový tensor, případně $\delta_{ij}$ Kronekerovo delta. Pro Newtonskou tekutinu lze tenzor viskózního smykového napětí vyjádřit podle \cite{hirsch2007numerical} jako 
\begin{equation}
\tau_{ij}=\mu \left[ \left( \dfrac{\partial u_j}{\partial x_i} + \dfrac{\partial u_i}{\partial x_j} \right) - \dfrac{2}{3} \delta_{ij} \left(\nabla \cdot \mathbf{u}\right)  \right],
\end{equation}
za předpokladu konstantní dynamické viskozity $\mu$ jak poukazuje \cite{dvorak1987vnitrniaerodynamika}.

Nyní lze již psát soustavu pohybových Navier-Stokesových (NS) rovnic v integrálním tvaru, tedy rovnice hybnosti pro stlačitelnou Newtonskou tekutinu, jako
\begin{equation}
\dfrac{\partial}{\partial t} \int_{\Omega} \rho \mathbf{u} \,\mathrm{d}V + \int_{\Gamma} \rho \mathbf{u} (\mathbf{u}\cdot \mathbf{n}) + p\mathbf{n} - \tau \cdot \mathbf{n} \,\mathrm{d}S = \int_\Omega \mathbf{Q_\Omega} \,\mathrm{d}V.
\end{equation}

Často lze rovnici hybnosti nalézt i v diferenciálním tvaru, například v \cite{hirsch2007numerical}
\begin{equation}
\rho \dfrac{\partial \mathbf{u}}{\partial t} + \rho (\mathbf{u} \cdot \nabla)\mathbf{u} +\nabla p - \mu \left[ \Delta \mathbf{u} + \dfrac{1}{3} \nabla(\nabla \cdot \mathbf{u}) \right] = \rho \mathbf{f_e}
\end{equation}

\subsection{NS rovnice pro nestlačitelnou tekutinu}

Obecný systém NS rovnic lze pro speciální případy zjednodušit zanedbáním některých fyzikálních vlivů. 
V této práci budeme později využívat zjednodušený tvar NS rovnic  pro nestlačitelnou tekutinu. 
Tedy $\rho=konst.$ čímž dostáváme rovnici kontinuity ve zjednodušeném tvaru \ref{eq:kontinuita_nestlacitelna}, tedy
\begin{equation}
\nabla\cdot\mathbf{u} = 0
\end{equation}
a NS rovnice hybnosti v diferenciálním tvaru podle \cite{hirsch2007numerical} má podobu 
\begin{equation}\label{eq:NS_icoDiff}
\rho \dfrac{\partial \mathbf{u}}{\partial t}+ \rho(\mathbf{u}\cdot \nabla)\mathbf{u} = -\nabla p + \mu \Delta \mathbf{u} + \rho \mathbf{f_e}.
\end{equation}
Rovnici hybnosti jde dále vydělit konstantou hustoty, čímž dostaneme jakýsi měrný tlak $ \widehat{p} = \dfrac{p}{\rho} $ a rovnice \ref{eq:NS_icoDiff} přejde do tvaru
\begin{equation}\label{eq:NS_icoPseudotlak}
\dfrac{\partial \mathbf{u}}{\partial t}+ (\mathbf{u}\cdot \nabla)\mathbf{u} = -\nabla \widehat{p} + \mu \Delta \mathbf{u} + \mathbf{f_e}.
\end{equation}



\section{Základ metody konečných objemů}
Metoda konečných objemů (MKO, anglicky Finite volume method - FVM) je jednou z nejpoužívanějších metod pro řešení PDR proudění - společně s konečnými diferencemi a metodou konečných prvků. 
Popularita MKO pro numerické řešení problému proudění tkví podle \cite{hirsch2007numerical} v její obecnosti, srozumitelnosti základních principů a snadnosti implementace pro libovolné sítě i složitější geometrie.

Zásadní výhodou z hlediska přesnosti MKO je pak princip tzv. konzervativní diskretizace (konzervativní ve smyslu zachovávající). Udržet v platnosti základní zákony zachování je důležitý aspekt správnosti řešení. MKO má tu výhodu, že konzervativní diskretizace je podle \cite{hirsch2007numerical} splněna automaticky díky přímé diskretizaci integrálního tvaru zákonů zachování. 

\subsection{Konečný objem}
MKO nese svůj název podle způsobu prostorové diskretizace, tj. rozdělení zkoumané oblasti $\Omega=\mathbb{R}^d$ na vzájemně disjunktní neprázdné otevřené podoblasti $\Omega_j$ s konečnou velikostí, matematicky psáno
\begin{align*}
\overline{\Omega} = \cup^n_{i=1}\overline{\Omega}_i,&\\
\Omega_i \cap \Omega_j = \emptyset,& \,\,\mathrm{pro} \, i \neq j.
\end{align*}
Tyto konečné objemy (někdy buňky) jsou analogií kontrolních objemů z podsekce \ref{sec:kontrolni_objem}. Jakmile máme takto rozdělenou výpočetní oblast, tak na každý konečný objem aplikujeme zákon zachování v integrálním tvaru. To si můžeme dovolit, neboť zákony zachování byly v sekci \ref{sec:zaklady_popisu} odvozeny pro libovolný kontrolní objem a lze je tedy aplikovat na každý konečný podobjem zvlášť. Obecný zákon zachování popsaný rovnicí \ref{eq:zachovani} má pro j-tý kontrolní objem tvar
\begin{equation}\label{eq:zachovani_MKO}
\frac{\partial}{\partial t} \int_{\Omega_j}W\,\mathrm{d}V + \int_{\Gamma_j} \mathbf{F} \cdot \mathbf{n} \,\mathrm{d}S = \int_{\Omega_j}Q_\Omega \,\mathrm{d}V,
\end{equation}
kde pro jednoduchost zápisu ponecháváme jen objemové zdroje na pravé straně.
Pro každý konečný objem nyní definujeme prostorově střední hodnotu sledované veličiny 
\begin{equation}
\overline{W}|_{\Omega_j}= W_j(t) = \frac{1}{|\Omega_j|}\int_{\Omega_j}W(\mathbf{x},t) \,\mathrm{d} V.
\end{equation}
Stejným způsobem nahradíme i objemové zdroje v rovnici \ref{eq:zachovani_MKO} a integrál toku $\mathbf{F}$ nahradíme součtem přes hranice. Dostaneme tvar rovnice zachování, napsanou pro j-tý kontrolní konečný objem
\begin{equation}\label{eq:polodiskretniMKO}
\frac{\partial}{\partial t} (W_j|\Omega_j|) + \sum_{\forall f} \int_{f}\mathbf{F}\cdot \mathbf{n} \,\mathrm{d}S = Q_j|\Omega_j|,
\end{equation} 
kde stěny $f$ jsou jednotlivé části hranice $\Gamma_j$ a všechny stěny tvoří vzájemně disjunktní pokrytí příslušné hranice.
Stojí za to podotknout, že rovnice \ref{eq:polodiskretniMKO} je stále matematicky ekvivalentní k rovnici \ref{eq:zachovani_MKO}.
Prozatím jsme ještě neprovedly žádné aproximace či přibližné náhrady.

\subsection{Aproximace numerickým tokem}

Nyní se pokusíme aproximovat integrál toku přes hranice z rovnice \ref{eq:polodiskretniMKO}. Pro lepší představu teď předpokládejme, že tok zachovávané veličiny je dán z rovnic \ref{eq:diff_tok} a \ref{eq:konv_tok} jako
\begin{equation}
\mathbf{F}=\mathbf{u}W-\kappa \nabla W.
\end{equation}
Tok přes stěnu $f$ (část hranice $\Gamma_j$) se souřadnicí středu $\mathbf{x_f}$ můžeme aproximovat pomocí
\begin{equation}\label{eq:aprox_tok}
\int_{f}\mathbf{F}\cdot \mathbf{n} \,\mathrm{d}S
=
\int_{f}(\mathbf{u}W-\kappa\nabla W)\cdot \mathbf{n}\, \mathrm{d}S 
\approx 
\left(\mathbf{u} W_f - \kappa \nabla W_f \right) \cdot \mathbf{S_f} 
= 
\mathbf{F_f} \cdot \mathbf{S_f},
\end{equation}
kde $\mathbf{S_f}=\int_{\Gamma_j}\mathbf{n}\,\mathrm{d}S$, což je konstantní vlastnost geometrie stěny, $W_f(t) = W(\mathbf{x_f},t)$ a $\nabla W_f(t) = \nabla W (\mathbf{x_f}, t)$.

Pro řešení úlohy je také potřeba zvolit, kde budou ukládány proměnné. Jinými slovy, jestli v našich rovnicích bude neznámá např. ve středu buňky (bude reprezentovat střední hodnotu v celé buňce) $W_j$, nebo uprostřed stěny $W_f$. V praxi se používá více možností i případných kombinací, jak uvádí \cite{blazek2015computational, hirsch2007numerical}.
Standardně se používá ukládání hodnot ve středu buněk, ve středu stěn či ve vrcholech.
V některých případech se objevuje i smíšený způsob (anglicky \textit{staggered}), kde hodnoty různých veličin jsou ukládány na jiných místech.
Dále budeme předpokládat, že proměnné uchováváme ve středu buněk (anglicky \textit{cell-centered}), tedy že proměnnou bude hodnota $W_j$.
Pro další postup je tedy potřeba aproximovat hodnoty $W_f$ a $\nabla W_f \cdot \mathbf{S_f}$ pomocí zavedených neznámých ve středech buněk a získat tak $ \mathbf{F_f} = \mathbf{F_f}(W_j)$.
Poté již můžeme napsat semidiskrétní tvar (ve smyslu MKO) rovnice zachování skalární veličiny 
\begin{equation}
\frac{\partial}{\partial t} (W_j|\Omega_j|) + \sum_{\forall f} \mathbf{F_f} \cdot \mathbf{S_f} = Q_j|\Omega_j|.
\end{equation}
Způsobů diskretizace numerického toku je mnoho, neboť jde o jednu ze stěžejních částí MKO. Numerický tok totiž zásadním způsobem ovlivňuje stabilitu a přesnost následného výpočtu. Dále jsou uvedeny pouze základní příklady způsobu diskretizace, neboť jejich rozbor není předmětem této práce.

\subsubsection{Diskretizace difuzivního toku}
Jak uvádí rovnice \ref{eq:aprox_tok}, aproximujeme člen difuzivního toku přes stěnu $f$ jako
\begin{equation}
-\int_f \kappa \nabla W \cdot \mathbf{n}\mathrm{d}S \approx \mathbf{F_D} = -\kappa \nabla W_f \cdot \mathbf{S_f}.
\end{equation}
Pro diskretizaci takového členu můžeme vztah upravit na
\begin{equation}
\mathbf{F_D} = -\kappa \dfrac{\partial W_f}{\partial \mathbf{n_f}} S_f,
\end{equation}
kde $\dfrac{\partial W_f}{\partial \mathbf{n_f}}$ je tzv. derivace ve směru normály stěny $f$ a $S_f$ je plocha stěny. Pokud stěna $f$ je právě mezi středy buněk $ j=C $ a $ j=N $, tedy $ n_f $ je vnější normála vzhledem k buňce $ C $ a vnitřní vzhledem k $ N $, tak lze derivaci ve směru aproximovat pomocí 
\begin{equation}
\dfrac{\partial W_f}{\partial \mathbf{n_f}} \approx \dfrac{W_N-W_C}{||\mathbf{x_N}-\mathbf{x_C}||}.
\end{equation}

\subsubsection{Diskretizace konvektivního toku}
Druhou částí toku přes střenu je konvektivní tok, z rovnice \ref{eq:aprox_tok} tedy
\begin{equation}
\int_f \mathbf{u}W\cdot \mathbf{n}\, \mathrm{d}S
\approx
\mathbf{F_K}
=
W_f \mathbf{u_f}\cdot \mathbf{S_f}=W_f\phi_f,
\end{equation}
kde jsme skalární součin $ \mathbf{u_f}\cdot \mathbf{S_f} $ označily jako $ \phi_f $, tzv. konvektivní tok přes stěnu $ f $. Interpolace $ W_f $ lze pro případ ortogonální sítě s konstantním krokem zapsat jednoduše jako
\begin{equation}
W_f = \dfrac{W_C+W_N}{2}.
\end{equation}
Interpolaci lze provést i lepšími způsoby, které kompenzují případné nedokonalosti či nerovnoměrnosti v síti. Například pro ortogonální síť s nerovnoměrným krokem je vhodnější formulovat interpolaci jako
\begin{equation}
W_f \approx \dfrac{||\mathbf{x_{Nf}}|| W_C + ||\mathbf{x_{Cf}}|| W_N }
{||\mathbf{x_{Nf}}|| + ||\mathbf{x_{Cf}}||}
= 
\dfrac{||\mathbf{x_{Nf}}|| W_C + ||\mathbf{x_{Cf}}|| W_N }
{||\mathbf{x_{CN}}||},
\end{equation}
kde $ \mathbf{x_{jf}} $ je vektor mezi středem buňky $ j=C,N $ a středem stěny $ f $.

\section{SIMPLE algoritmus}


myslenka, 'podvod', relaxace

Algoritmus SIMPLE (zkratka pro \textit{semi-implicit pressure linked equations}) pro řešení problému proudění nestlačitelné tekutiny lze v jeho původní variantě nalézt například v \cite{patankar1983calculation}. Od té doby se objevilo spoustu úprav a vylepšení jako SIMPLER, SIMPLEST nebo SIMPLEC. 

\subsection{Myšlenka segregovaných algoritmů}
Algoritmus SIMPLE je jeden ze základních příkladů tzv. segregovaných algoritmů. Rovnice ze soustavy NS rovnic se zde neřeší jako jeden celek, ale odděleně každá zvlášť. Výhodou oproti klasickému sdruženému algoritmu je, že se vyhneme řešení rozsáhlé soustavy rovnic se špatně podmíněnou maticí, jak poukazuje \cite{furst2020mko2}. 

Segregované algoritmy obecně naráží na problémy s konvergencí či přesností jakmile se zvýší závislost mezi jednotlivými rovnicemi soustavy. Jinými slovy matice soustavy sestavená sdruženou metodou začne být lépe podmíněná. V případě soustavy NS rovnic pro tekutinu o konstantní hustotě může být měřítkem fiktivní Machovo číslo $ M = \dfrac{||\mathbf{u}||}{c} $, kde $ c $ má význam klidové rychlosti zvuku v tekutině.

\subsection{Varianta algoritmu SIMPLE s rovnicí pro tlak}
V softwarové knihovně OpenFOAM \cite{weller1998tensorial}, která je pro potřeby aplikace v rámci této práce využita, je dle \cite{furst2020mko2} algoritmus SIMPLE implementován v následující formě. 

Nejprve se stanoví odhad rychlosti $ \mathbf{u}^* $ pomocí tlak z předchozí iterace (případně z počáteční podmínky) z diskretizované rovnice hybnosti \ref{eq:NS_icoPseudotlak}, ve které je konvektivní člen linearizován Picardovou aproximací. Pro odhad rychlosti tak dostáváme rovnici
\begin{equation}
a_C^0\mathbf{u_C^*}
=
\sum_f a_{CN}^0 \mathbf{u_N^*}+\mathbf{Q_C^0}-\nabla p_c^0.
\end{equation}
Zde horní indexy označují iteraci, tedy index $ \null^0 $ předchozí iteraci, index $ \null^* $ odhad hodnoty nové iterace a později $ \null^n $ hodnotu v nové iteraci. Dolní indexy pak označují buňku $ C $, která sdílí stěnu $ f $ se sousední buňkou $ N $. Koeficienty $ a $ jsou určeny podle metody diskretizace jednotlivých členů rovnice.

Označíme část rovnice pro odhad rychlosti
\begin{equation}\label{key}
 \dfrac{1}{a_C^0} (\sum_f a_{CN}^0 \mathbf{u_N^*}+\mathbf{Q_C^0}) = \mathbf{\widehat{u}^*_C}
\end{equation}
a po interpolaci na stěny

2

1 urceni odhadu rychlosti ze stareho tlaku, starych zdrojovych clenu a koeficientu $ a_C $ (relaxace rovnice) 3.86
2 urceni odhadu tlaku z noveho odhadu rychlosti 3.90
3 ziskani novych hodnot rychlosti a toku $ \phi_f $ 3.91,3.92
4 vypocet noveho tlaku (relaxace korekce) 3.93

\section{Turbulence, modelování turbulence}
v rychlosti o RANS, model turbulence Spalart a K-Omega rozdil vyhody a vhodnost aplikace (ucel modelu)


\chapter{Optimalizace sdruženou metodou}
%!TEX ROOT=../../_main.tex
Zájem o optimalizaci proudění tekutin je od nepaměti a předmětem vědeckého bádání minimálně od doby vynalezení integrálního počtu \cite{karman1997inverse}. Tato kapitola se zabývá základní definicí problému optimalizace a představuje známou, avšak v oblasti proudění tekutiny, prozatím nepříliš hojně užívanou metodu optimalizace. Dále jsou odvozeny základy této metody pro její aplikaci v druhé části této práce.
\section{Základy optimalizace}\label{sec:zaklady_opt}
Pro popis problému optimalizace se používají pojmy:
\begin{itemize}
	\item \textit{stavové proměnné} $ \phi $ nebo také fyzikální veličiny či proměnné jako tlak, rychlost, teplota atd. dané většinou z konkrétních rovnic
	\item \textit{návrhové parametry} $ g $ materiálové vlastnosti, vstupní rychlost, tvar geometrie nebo hranice
	\item \textit{cílová funkce/funkcionál} $ J(\phi,g) $ hodnocení kombinace stavových a návrhových parametrů, např. tlaková ztráta, stlačení nebo vztlak
	\item \textit{vazební rovnice} $ R(\phi,g)=0 $ rovnice proudění
\end{itemize}
Problém optimalizace lze pak matematicky formulovat následovně. \cite{karman1997inverse}
\begin{problem}\label{prob:optimalizace}
Nechť je dána množina parametrů $ g=\left\lbrace g_n, \, n=1,...,N\right\rbrace $, cílová funkce $ J(\phi, g) $ a vazební rovnice $ R(\phi, g)=0 $. 
Najděte takovou kombinaci parametrů $ g $ a $ \phi $, která minimalizuje funkci $ J(\phi, g) $ a zároveň splňuje platnost podmiňujících rovnic $ R(\phi, g)=0$ . 
\end{problem}
\nomenclature[B]{$ b_n $}{Optimalizační parametry}
\nomenclature[J]{$ J $}{Cílová funkce}
\nomenclature[R]{$ R_i $}{Podmiňující rovnice}
Metod na řešení optimalizačního problému je hned několik.
Obecně je lze rozlišit na obecné a lokální optimalizační metody. Mezi všeobecně známe patři například metoda genetických algoritmu(obecná) a gradientní optimalizační metody(lokální). 
Základní rozdíl těchto metod je, že obecné optimalizační metody se zpravidla snaží přiblížit globálnímu optimu v celém prostoru přípustných parametrů, kdežto lokální metody na základě počátečního odhadu spadají do nejbližšího lokálního minima.

Typický optimalizační cyklus lokální metody lze zapsat následovně:
\begin{addmargin}[3em]{2em}% 1em left, 2em right
	\begin{description}
	\item 	Mějme počáteční odhad $ g^{(0)} $
	\item 	Pro $ n=0,1,2... $
	 	\begin{enumerate}
			\item Vyřešit $ R(\phi^{(n)},g^{(n)}) $ pro zjištění $ \phi^{(n)} $
			\item Spočítat $ \dfrac{\mathrm{d}J}{\mathrm{d}g}(\phi^{(n)},g^{(n)}) $
			\item Pomocí výsledků 1 a 2 zjistit optimální krok $ \delta g $ - např. $ \delta g = -\alpha \dfrac{\mathrm{d}J}{\mathrm{d}g}(\phi^{(n)},g^{(n)}) $
			\item Změnit návrhové proměnné $ g^{(n+1)} = g^{(n)} + \delta g $
		\end{enumerate}
	\end{description}
\end{addmargin}
Různé algoritmy se odlišují ve způsobu vyhodnocení gradientu v kroku 2. (citlivostní gradient, sdružená metoda) a následně se větví při volbě vhodného optimalizačního kroku (CGM, BFSG).  

\section{Metoda sdružené optimalizace}

Metoda sdružené optimalizace se snaží vyřešit problém popsaný v sekci \ref{sec:zaklady_opt}. 
Jde o speciální případ gradientní metody optimalizace, a tedy se předpokládá, že původní výběr optimalizovaných parametrů se nachází poměrně blízko hledaného optima. Nové, optimálnější řešení se dostane podle předpisu
\begin{equation}\label{eq:bnew_step}
g^{(n+1)}=g^{(n)}-\alpha\cdot\dfrac{\mathrm{d}J}{\mathrm{d}g}(\phi^{(n)}, \, g^{(n)}),
\end{equation}
kde $ \alpha < 0 $ je délka kroku. 
Co se týče znaménka v rovnici \ref{eq:bnew_step}, tak to je v tomto případe $ - $, neboť dle problému \ref{prob:optimalizace} hledáme minimum funkcionálu $ J $ a tedy musíme dělat krok proti směru nejvyššího růstu i.e. ve směru opačném ke gradientu.

Hlavním znakem sdružené gradientní optimalizace je způsob vyhodnocení gradientu cílové funkce vzhledem parametrům, tedy $ \frac{\mathrm{d}J}{\mathrm{d}g} $. 
Pro vyhodnocení tohoto gradientu jsou odvozeny nové parciální diferenciální rovnice (PDR).
\nomenclature[P]{PDR}{Parciální diferenciální rovnice}
Proces odvození nových PDR z metody Lagrangeových multiplikátorů, která specifikuje novou cílovou funkci, která v sobě bude zahrnovat podmiňující rovnice. 
Definujeme tak novou cílovou funkci 
\begin{equation}\label{eq:L_funkcional}
L(\phi, g,\,\xi) = J(\phi, g) + \left\langle R(\phi, g),\xi \right\rangle,
\end{equation}
kde $ \xi $ jsou tzv. sdružené proměnné (sdružené ke stavovým proměnným) a $  \left\langle \, \cdot\,,\cdot \,  \right\rangle $ je symetrická, bilineární forma, jejíž podoba je zpravidla jasná až z konkrétně řešeného problému.
Dostáváme tak nový problém, jehož řešení je však podle \cite{karman1997inverse} ekvivalentní s problémem \ref{prob:optimalizace}.

\begin{problem}\label{prob:Lagrange}
Nechť je dána množina parametrů $ g=\left\lbrace g_n, \, n=1,...,N\right\rbrace $, cílová funkce $ J(\phi, g) $ a vazební rovnice $ R(\phi, g)=0 $.
Najděte takovou kombinaci parametrů g, stavových proměnných $ \phi $ a sdružených proměnných $ \xi $ tak, aby $ L(\phi, g,\,\xi) = J(\phi, g) + \left\langle R(\phi, g),\xi \right\rangle$ bylo  stacionární.
\end{problem}

Z matematického hlediska je dobré podotknout, že všechny argumenty $ L $ jsou na sobě nezávislé. Pro $ J $ tomu tak nebylo, protože $ \xi $ a $ g $ spolu byli svázané přes podmiňující rovnice $ R(\phi, g)=0 $ a nešlo je tak volit nezávisle. Abychom splnili podmínku stacionarity, jak požaduje problém \ref{prob:Lagrange}, musí být variance $ L $ podle všech proměnných rovna nule.

\subsection{Optimální systém rovnic}
V této části jsou ukázány jednotlivé variace 
\begin{equation}\label{eq:L_variace}
\delta L = \delta_\xi L + \delta_\phi L + \delta_g L
\end{equation} 
pro cílovou funkci $ L $ definovanou rovnicí \ref{eq:L_funkcional}.
\subsubsection{Variace sdružených proměnných}
Nulovost variace podle sdružených proměnných $ \xi $ lze zapsat jako
\begin{equation*}
\delta_\xi L = \lim\limits_{\epsilon\rightarrow0}\dfrac{L(\phi,g,\xi+\epsilon\delta\xi)-L(\phi,g,\xi)}{\epsilon}=0,
\end{equation*}
kde variace $ \delta\xi $ je libovolná. Po dosazení za $ L $ z rovnice \ref{eq:L_funkcional}, dostáváme
\begin{equation*}
\lim\limits_{\epsilon\rightarrow0} \dfrac
{J(\phi, g) +  \left\langle\xi+\epsilon\delta\xi, R(\phi, g)\right\rangle - (J(\phi, g) +   \left\langle\xi , R(\phi, g))\right\rangle }
{\epsilon}
=0,
\end{equation*}
tedy 
\begin{equation*}
 \left\langle\delta\xi , R(\phi, g) \right\rangle = 0.
\end{equation*}
Díky libovolnosti variace $ \delta\xi $ dostáváme původní vazební rovnici 
\begin{equation}\label{eq:vazebni_rce}
R(\phi, g)=0,
\end{equation}
která tvoří první část systému optimálních rovnic. Variace podle sdružených proměnných nám z části ukázala, že stacionární bod rozšířeného funkcionálu $ L(\phi, g,\,\xi)  $ splňuje vazební rovnice.

\subsubsection{Variace stavových proměnných}
Dále vezměme variaci vzhledem ke stavovým proměnným $ \phi $, tedy
\begin{equation*}
\delta_\phi L =
\lim\limits_{\epsilon\rightarrow0}
\dfrac{L(\phi+\epsilon\delta\phi,g,\xi)-L(\phi,g,\xi)}
{\epsilon}
=0
\end{equation*}
a opět po dosazení rovnice \ref{eq:L_funkcional}
\begin{equation*}
\lim\limits_{\epsilon\rightarrow0} \dfrac
{J(\phi+\epsilon\delta\phi, g) + 
 \left\langle\xi , R(\phi+\epsilon\delta\phi, g) \right\rangle -  (J(\phi, g) +  \left\langle\xi , R(\phi, g)\right\rangle)}
{\epsilon}
=0,
\end{equation*}
neboli
\begin{equation*}
\lim\limits_{\epsilon\rightarrow 0} 
\left(
\dfrac
{J(\phi+\epsilon\delta\phi, g) - J(\phi, g)}
{\epsilon}
+
\dfrac
{  \left\langle\xi , R(\phi+\epsilon\delta\phi) - R(\phi, g)\right\rangle }
{\epsilon}
\right)
=0.
\end{equation*}
Členy ve jmenovateli obsahující $ \epsilon $ přepíšeme pomocí Taylorova rozvoje okolo bodu $ \phi $
\begin{equation*}
\lim\limits_{\epsilon\rightarrow 0} 
\left(
\dfrac
{J(\phi, g) + \dfrac{\partial J}{\partial \phi}\epsilon\delta\phi + O(e^2) - J(\phi, g)}
{\epsilon}
+
\dfrac
{  \left\langle\xi , R(\phi,g) + \dfrac{\partial R}{\partial \phi} \epsilon \delta\phi + O(e^2) - R(\phi,g)\right\rangle }
{\epsilon}
\right)
=0,
\end{equation*}
kde vypadnou členy bez derivace, a po zkrácení $ \epsilon $ dostáváme
\begin{equation*}
\lim\limits_{\epsilon\rightarrow 0} 
\left(
\dfrac{\partial J}{\partial \phi}\delta\phi
+  \left\langle\xi , \dfrac{\partial R}{\partial \phi} \delta\phi \right\rangle
+O(\epsilon)
\right)
=0.
\end{equation*}
Provedeme limitu 
\begin{equation}\label{eq:sdruzene_rce_variace}
\frac{\partial J}{\partial \phi}\delta\phi
+  \left\langle\xi , \dfrac{\partial R}{\partial \phi} \delta\phi\right\rangle
=0,
\end{equation}
použijeme 
\begin{equation*}
\frac{\partial J}{\partial \phi}\delta\phi = \left\langle 1, \frac{\partial J}{\partial \phi}\delta\phi \right\rangle
\end{equation*}
a označíme $ (\cdot)^* $ sdružený operátor k $  \left\langle \, \cdot\,,\cdot \,  \right\rangle $, tedy
\begin{equation*}
\left\langle   \left(\frac{\partial J}{\partial \phi}\right)^* ,\delta\phi  \right\rangle
+  \left\langle \left(\dfrac{\partial R}{\partial \phi}\right)^* \xi ,  \delta\phi\right\rangle
=0.
\end{equation*}
Další část systému optimálních rovnic jsou tedy tzv. sdružené rovnice
\begin{equation}\label{eq:sdruzene_rce}
\left( \dfrac{\partial R}{\partial \phi} \right)^* \xi = 
- \left(\dfrac{\partial J}{\partial \phi}\right)^*.
\end{equation}

\subsubsection{Variace návrhových parametrů}

Poslední část variace $ \delta L $ je vzhledem k návrhovým parametrům $ g $, tedy
\begin{equation*}
\delta_g L =
\lim\limits_{\epsilon\rightarrow 0}
\dfrac{L(\phi,g+\epsilon\delta g,\xi)-L(\phi,g,\xi)}
{\epsilon}
=0,
\end{equation*}
kde obdobně jako pro stavové proměnné po dosazení z rovnice \ref{eq:L_funkcional} dostaneme
\begin{equation*}
\lim\limits_{\epsilon\rightarrow0} \dfrac
{J(\phi, g+\epsilon\delta g) + 
	 \left\langle\xi , R(\phi, g+\epsilon\delta g)\right\rangle  -  (J(\phi, g) +  \left\langle\xi , R(\phi, g)\right\rangle)}
{\epsilon}
=0
\end{equation*}
a po aplikování stejného postupu jako pro vztah \ref{eq:sdruzene_rce}, dostaneme poslední část optimálního systému rovnic, tzv. podmínky optimálnosti
\begin{equation}\label{eq:podminky_optimalnosti}
\left( \dfrac{\partial R}{\partial g} \right)^* \xi = 
- \left(\dfrac{\partial J}{\partial g}\right)^*.
\end{equation}

\subsubsection{Řešení soustavy optimálních rovnic}
Řešením soustavy optimálních rovnic \ref{eq:vazebni_rce}, \ref{eq:sdruzene_rce} a \ref{eq:podminky_optimalnosti} dává řešení problému \ref{prob:Lagrange} a tedy i \ref{prob:optimalizace}. Analyticky lze systém vyřešit pouze ve speciálních případech a oproti základnímu systému, tedy vazebním rovnicím, je tento nesegregovaný systém často masivní, jak upozorňuje \cite{karman1997inverse}. Vyřešením této soustavy jako nesegregované dostaneme přímo optimální hodnoty návrhových parametrů, bohužel to v mnoha případech není možné. Řešení soustavy rovnic segregovaným způsobem už je schůdnější varianta, vyžaduje však iteraci a lze ukázat, že iterační metoda řešící každou z tří části odděleně je ekvivalentní k metodě nejvyššího spádu, jejíž rychlost konvergence je často nedostačující. Jak bylo již řečeno v úvodu této sekce, lze sdruženou metodu použít i jiným způsobem a to pro výpočet gradientu/variace $ \frac{\mathrm{d} J}{\mathrm{d} g} $, který je posléze použit v optimalizačním cyklu podle předpisu \ref{eq:bnew_step}.

\subsection{Gradient pomocí sdružené metody}
Získat gradient cílové funkce je v rámci optimalizačního cyklu, nastíněného na začátku této sekce, tradičně nejnáročnější operace. Gradient cílové funkce lze za použití řetízkového pravidla zapsat jako
\begin{equation}\label{eq:gradient_cilove_fce}
\dfrac
{\mathrm{d}J}
{\mathrm{d}g} (\phi^{(n)}, \, g^{(n)})
=
\dfrac
{\partial J}
{\partial \phi }(\phi^{(n)}, \, g^{(n)})
\dfrac
{\mathrm{d}\phi}
{\mathrm{d}g}(\phi^{(n)}, \, g^{(n)})
+
\dfrac
{\partial J}
{\partial g}(\phi^{(n)}, \, g^{(n)}).
\end{equation}
Mějme řešení sdružených rovnic \ref{eq:sdruzene_rce} v n-té iteraci, i.e. mějme $ \xi^{(n)} $, které řeší
\begin{equation}\label{eq:temp1}
\xi^{(n)} \cdot \dfrac{\partial R}{\partial \phi} \Bigr|_{g^{(n)}}=\dfrac{\partial J}{\partial \phi}\Bigr|_{g^{(n)}}.
\end{equation}
Dosazením \ref{eq:temp1} do rovnice \ref{eq:gradient_cilove_fce} dostaneme 
\begin{equation*}
\dfrac{\mathrm{d}J}
{\mathrm{d}g} (\phi^{(n)}, \, g^{(n)})
=
\xi^{(n)} 
\cdot 
\left(
\dfrac{\partial R}
{\partial \phi}\Bigr|_{g^{(n)}}
\right)
\dfrac{\mathrm{d}\phi}
{\mathrm{d}g}\Bigr|_{g^{(n)}}
+
\dfrac{\partial J}
{\partial g}\Bigr|_{g^{(n)}},
\end{equation*}
kam dosadíme variaci vazebních rovnic $ R(\phi,g)=0 $
\begin{equation*}
\left(
\dfrac{\partial R}
{\partial \phi}\Bigr|_{g^{(n)}}
\right)
\dfrac{\mathrm{d}\phi}
{\mathrm{d}g}\Bigr|_{g^{(n)}}
=
-\dfrac{\partial R}
{\partial g}\Bigr|_{g^{(n)}},
\end{equation*}
čímž se dopracujeme k hledanému vztahu
\begin{equation*}
\dfrac{\mathrm{d}J}
{\mathrm{d}g} (\phi^{(n)}, \, g^{(n)})
=
-\xi^{(n)} 
\cdot 
\dfrac{\partial R}
{\partial g}\Bigr|_{g^{(n)}}
+
\dfrac{\partial J}
{\partial g}\Bigr|_{g^{(n)}},
\end{equation*}
V optimalizačních úlohách se k zjištění gradientů typicky používá metoda konečných diferencí. Náročnost výpočtu ale roste přímo úměrně s počtem návrhových parametrů, kdežto ve sdružené metodě závisí vyhodnocení gradientu převážně na rychlosti výpočtu nových PDR. Ve sdružené metodě je pak navíc potřeba odvodit příslušné PDR a to teoreticky pro každou cílovou funkci. Prakticky lze ale určit dostatečně obecný předpis pro cílovou funkci, odvodit rovnice pro ni a konkrétní předpis dosadit až později.

\subsection{Sdružené rovnice pro nestlačitelné proudění}

V této podsekci budou odvozeny sdružené rovnice pro tvarovou optimalizaci v nestlačitelném proudění podle \cite{papadimitriou2007continuous, furst2020mko2}. Podmiňující rovnice budou tedy NS rovnice pro stacionární stav s tekutinou o konstantní hustotě, tedy rovnice hybnosti
\begin{equation}\label{eq:rce_hybnosti}
\mathbf{R^u}=\left(\mathbf{u}\cdot\nabla\right)\mathbf{u} + \nabla p - \nabla \cdot 2 \nu D(\mathbf{u}),
\end{equation}
kde operátor $ D(\mathbf{u})=\frac{1}{2}(\nabla\mathbf{u}+\nabla\mathbf{u}^T) $, a rovnice kontinuity
\begin{equation}
R^p=-\nabla \cdot \mathbf{u}.
\end{equation}\label{eq:rce_kontinuity}
Stavové proměnné jsou tedy v tomto případě rychlost $ \mathbf{u} $ a měrný tlak ve smyslu rovnice \ref{eq:NS_icoPseudotlak}, tedy $ p = \dfrac{p_{fyz}}{\rho}$. K nim budou odpovídat sdružené proměnné $ \xi = (\mathbf{v},q) $. Obecnou cílovou funkci lze v rámci mnoha inženýrských aplikací zapsat pomocí integrálu přes výpočetní oblast a integrálu přes hranici, tedy
\begin{equation}\label{eq:cenova_fce}
J(\mathbf{u},p,g)=\int_{\Omega} J_\Omega(\mathbf{u},p,g) \, \mathrm{d}V + \int_{\Gamma}J_\Gamma(\mathbf{u},p,g) \, \mathrm{d}S.
\end{equation}
S takto definovanými proměnnými, vazebními rovnicemi a cenovou funkcí lze definovat upravenou cenovou funkci
\begin{equation}
L= J + \int_\Omega \xi \cdot \mathbf{R} \,\mathrm{d}\Omega 
%= J + \int_\Omega \mathbf{v}\cdot\mathbf{R^u}+p\, R^p \,\mathrm{d}\Omega
= \int_\Omega \mathbf{v}\cdot\mathbf{R^u}+ q R^p +J_\Omega  \,\mathrm{d}\Omega + \int_{\Gamma}J_\Gamma \, \mathrm{d}S
\end{equation}
a pokračovat k odvození sdružených rovnic a definujeme vhodnou bilineární formu 
\begin{equation}
\left\langle \mathbf{a},\mathbf{b} \right\rangle = \int_{\Omega} \mathbf{a} \cdot \mathbf{b} \, \mathrm{d}V.
\end{equation} 
Vyjdeme předpisu \ref{eq:sdruzene_rce_variace} pro variaci upravené cenové funkce podle stavových proměnných
\begin{equation} \label{eq:sdruzena_variace}
\delta_\phi L = 
\frac{\partial J}{\partial \phi}\delta\phi
+
\int_{\Omega} 
\xi \cdot \dfrac{\partial R}{\partial \phi}  \delta\phi 
\, \mathrm{d}V
 = 0
\end{equation}
Pro člen s $ J $ můžeme přímo psát
\begin{align}
\dfrac{\partial J}{\partial u_i}& \delta u_i = \int_{\Omega} \dfrac{\partial J_{\Omega}}{\partial u_i} \delta u_i \, \mathrm{d}V + \int_{\Gamma} \dfrac{\partial J_{\Gamma}}{\partial u_i} \delta u_i \, \mathrm{d}S, \\
\dfrac{\partial J}{\partial p} &\delta p \,\,= \int_{\Omega} \dfrac{\partial J_{\Omega}}{\partial p} \delta p  \, \mathrm{d}V + \int_{\Gamma} \dfrac{\partial J_{\Gamma}}{\partial p} \delta p  \, \mathrm{d}S.
\end{align}
Pro člen se sdruženými proměnnými $ \xi \cdot \dfrac{\partial R}{\partial \phi} $ provedeme variaci $ \delta\phi $ postupně pro každou rovnici
\begin{gather*}
\delta_\mathbf{u} \mathbf{R^u}=
(\delta \mathbf{u}\cdot \nabla )\mathbf{u} + (\mathbf{u}\cdot \nabla )\delta\mathbf{u} - \nabla \cdot (2\nu D(\delta \mathbf{u}) ) ,\\
\delta_p \mathbf{R^u}= \nabla \delta p,\\
\delta_\mathbf{u}R^p = -\nabla \cdot \delta \mathbf{u} ,\\
\delta_p R^p = 0.
\end{gather*}
Pro úplnost je potřeba podotknout, že jsme vynechali variaci kinematické viskozity $ \nu $. 
Pro laminární proudění je to naprosto platný postup. 
Pokud ale budeme modelovat turbulenci za pomocí přídavné turbulentní vazkosti $ \nu = \nu_t + \widetilde{\nu} $, tak buďto musíme udělat předpoklad tzv. zmražené turbulence (anglicky frozen turbulence) a nebo rozepsat i variace rovnic turbulence. Zkoumání variace modelů turbulence je nad rámec této práce a detailněji se o něm píše v \cite{zymaris2009continuous}, kde se rozebírá i vliv zjednodušujícího předpokladu zmrazené turbulence na výsledek pro případ různých Reynoldsových čísel. 

Odvozené vztahy nyní dosadíme do předpisu \ref{eq:sdruzena_variace}
\begin{multline}
\delta_\phi L = 
\int_{\Omega} \dfrac{\partial J_{\Omega}}{\partial u_i} \delta u_i \, \mathrm{d}V 
+ 
\int_{\Gamma} \dfrac{\partial J_{\Gamma}}{\partial u_i} \delta u_i \, \mathrm{d}S
+
\int_{\Omega} \dfrac{\partial J_{\Omega}}{\partial p} \delta p  \, \mathrm{d}V 
+ 
\int_{\Gamma} \dfrac{\partial J_{\Gamma}}{\partial p} \delta p  \, \mathrm{d}S
+\\+
\int_{\Omega} 
		\underbrace{\mathbf{v} \cdot(\delta \mathbf{u}\cdot \nabla )\mathbf{u}}_\mathrm{I}
		+ \underbrace{\mathbf{v} \cdot(\mathbf{u}\cdot \nabla )\delta\mathbf{u}}_\mathrm{II}
		\underbrace{ - \mathbf{v} \cdot (\nabla \cdot (2\nu D(\delta \mathbf{u}) ))}_\mathrm{III}
\, \mathrm{d}V
+\\+
\underbrace{
\int_{\Omega} 
 \mathbf{v} \cdot \nabla \delta p
\, \mathrm{d}V
}_\mathrm{IV}
+
\underbrace{
\int_{\Omega} 
  - q \nabla \cdot \delta \mathbf{u}
\, \mathrm{d}V.
}_\mathrm{V} 
\end{multline}
Postupně teď pomocí Gaussovy věty (integrace per-partes) přesuneme derivace na sdružené proměnné, tak abychom mohli vytknout variace stavových proměnných. Pro odvození některých členů je výhodnější použít indexový zápis s Einstainovou sumační konvencí, tedy
\begin{equation*}
\mathrm{I}
=
\int_{\Omega} 
\mathbf{v}\cdot(\delta\mathbf{u}\cdot \nabla)\mathbf{u}
\, \mathrm{d}V
=
\int_{\Omega} 
v_j \delta u_i \frac{\partial u_j}{x_i}
\, \mathrm{d}V
\end{equation*}
a po aplikaci Gaussovy věty
\begin{equation*}
\mathrm{I}
=
\int_{\Gamma} 
v_j \delta u_i u_j n_i
\, \mathrm{d}S
-
\int_{\Omega} 
\frac{\partial( v_j \delta u_i )}{x_i}u_j
\, \mathrm{d}V,
\end{equation*}
rozepíšeme derivaci součinu
\begin{equation*}
\mathrm{I}
=
\int_{\Gamma} 
v_j \delta u_i u_j n_i
\, \mathrm{d}S
-
\int_{\Omega} 
v_j\underbrace{\frac{\partial( \delta u_i )}{x_i}u_j}_{=0 \text{ z kontinuity} }
\, \mathrm{d}V,
-
\int_{\Omega} 
\delta u_i \frac{\partial v_j  }{x_i}u_j
\, \mathrm{d}V
\end{equation*} 
a tedy
\begin{equation}\label{eq:clen_I}
\mathrm{I}
=
\int_{\Gamma} 
v_j u_j n_i \delta u_i 
\, \mathrm{d}S
-
\int_{\Omega} 
\frac{\partial v_j  }{x_i} u_j \delta u_i
\, \mathrm{d}V
=
\int_{\Gamma} 
(\mathbf{v}\cdot \mathbf{u} )\mathbf{n} \cdot \delta\mathbf{u}
\, \mathrm{d}S
-
\int_{\Omega} 
(\nabla \mathbf{v}\cdot \mathbf{u})\cdot\delta \mathbf{u}
\, \mathrm{d}V.
\end{equation}
Pro druhý člen můžeme psát
\begin{equation*}
\mathrm{II}
=
\int_{\Omega} 
\mathbf{v}\cdot(\mathbf{u}\cdot \nabla)\delta\mathbf{u}
\, \mathrm{d}V
=
\int_{\Omega} 
v_j  u_i \frac{\partial (\delta u_j)}{x_i}
\, \mathrm{d}V
\end{equation*}
a stejným postupem jako v předchozím případě (Gaussova věta, derivace součinu, rovnice kontinuity) dostáváme
\begin{equation}\label{eq:clen_II}
\mathrm{II}
=
\int_{\Gamma} 
v_j u_i n_i \delta u_j  
\, \mathrm{d}S
-
\int_{\Omega} 
u_i \frac{\partial v_j  }{x_i} \delta u_j
\, \mathrm{d}V
=
\int_{\Gamma} 
(\mathbf{n} \cdot \mathbf{u}) \mathbf{v}\cdot \delta \mathbf{u} 
\, \mathrm{d}S
-
\int_{\Omega} 
(\mathbf{u} \cdot \nabla)\mathbf{v}\cdot \delta \mathbf{u}
\, \mathrm{d}V.
\end{equation}
Nejpracnější je pak člen číslo tři, tedy
\begin{equation*}
\mathrm{III}
=
\int_{\Omega} 
-\mathbf{v}\cdot \left(\nabla \cdot (2\nu D(\delta \mathbf{u}) )\right)
\, \mathrm{d}V
=
\int_{\Omega} 
- v_i \frac{\partial}{\partial x_j} \left(  \nu
\left(\frac{\partial (\delta \mathbf{u}_i)}{\partial x_j} + 
\frac{\partial (\delta \mathbf{u}_j)}{\partial x_i}\right)
\right)
\, \mathrm{d}V,
\end{equation*}
použijeme Gaussovu větu poprvé
\begin{equation*}
\mathrm{III}
=
\int_{\Gamma} 
- v_i \left(  \nu
\left(\frac{\partial (\delta \mathbf{u}_i)}{\partial x_j} + 
\frac{\partial (\delta \mathbf{u}_j)}{\partial x_i}\right)
\right) n_j
\, \mathrm{d}S
+
\int_{\Omega} 
\frac{\partial v_i}{\partial x_j}   \nu
\left(\frac{\partial (\delta \mathbf{u}_i)}{\partial x_j} + 
\frac{\partial (\delta \mathbf{u}_j)}{\partial x_i}\right)
\, \mathrm{d}V
\end{equation*}
a podruhé, zvlášť na oba členy v objemovém integrálu,
\begin{multline*}
\mathrm{III}
=
\int_{\Gamma} 
- 2\nu \mathbf{v} \cdot  D(\delta \mathbf{u})\cdot \mathbf{n}
\, \mathrm{d}S
+
\int_{\Gamma} 
\nu \frac{\partial v_i}{\partial x_j}
(n_j \delta u_i + n_i \delta u_j)
\, \mathrm{d}S
-\\-
\int_{\Omega} 
\frac{\partial}{\partial x_j}\left( \nu \frac{\partial v_i}{\partial x_j} \right) \delta u_i + \frac{\partial}{\partial x_i}\left( \nu \frac{\partial v_i}{\partial x_j} \right) \delta u_j
\, \mathrm{d}V,
\end{multline*}
čímž dostáváme výraz
\begin{equation}\label{eq:clen_III}
\mathrm{III}
=
\int_{\Gamma} 
2\nu \mathbf{n} \cdot  D(\mathbf{v})\cdot \delta \mathbf{u}
\, \mathrm{d}S
- \int_{\Gamma} 
2\nu \mathbf{v} \cdot  D(\delta \mathbf{u})\cdot \mathbf{n}
\, \mathrm{d}S
-
\int_{\Omega} 
\nabla \cdot \left( 2\nu D(\mathbf{v}) \right) \cdot \delta \mathbf{u}
\, \mathrm{d}V.
\end{equation}
Dále
\begin{equation*}
\mathrm{IV}
=
\int_{\Omega} 
\mathbf{v} \cdot \nabla \delta p
\, \mathrm{d}V
=
\int_{\Omega} 
v_i \frac{\partial (\delta p) }{\partial x_i}
\, \mathrm{d}V
=
\int_{\Gamma} 
v_i n_i \delta p 
\, \mathrm{d}S
-
\int_{\Omega} 
 \frac{\partial v_i }{\partial x_i} \delta p
\, \mathrm{d}V
\end{equation*}
a 
\begin{equation}\label{eq:clen_IV}
\mathrm{IV}
=
\int_{\Gamma} 
\mathbf{v}\cdot \mathbf{n} \delta p 
\, \mathrm{d}S
-
\int_{\Omega} 
\nabla \cdot \mathbf{v} \delta p
\, \mathrm{d}V.
\end{equation}
Pro poslední člen pak
\begin{equation*}
\mathrm{V}
=
\int_{\Omega} 
- q \nabla \cdot \delta \mathbf{u}
\, \mathrm{d}V
=
\int_{\Omega} 
- q \frac{\partial (\delta u_i)}{\partial x_i}
\, \mathrm{d}V
= 
\int_{\Gamma} 
- q \delta u_i n_i
\, \mathrm{d}S
+
\int_{\Omega} 
 \frac{\partial q}{\partial x_i} \delta u_i
\, \mathrm{d}V
\end{equation*}
a vektorově tedy
\begin{equation}\label{eq:clen_V}
\mathrm{V}
=
\int_{\Gamma} 
- q \delta \mathbf{u \cdot n}
\, \mathrm{d}S
+
\int_{\Omega} 
\nabla q \cdot \delta \mathbf{u}
\, \mathrm{d}V.
\end{equation}

Konečně lze tedy rozepsat rovnici \ref{eq:sdruzena_variace} pomocí odvozených členů
\begin{multline*}
\delta_\phi L = 
\int_{\Omega} \dfrac{\partial J_{\Omega}}{\partial u_i} \delta u_i \, \mathrm{d}V 
+ 
\int_{\Gamma} \dfrac{\partial J_{\Gamma}}{\partial u_i} \delta u_i \, \mathrm{d}S
+
\int_{\Omega} \dfrac{\partial J_{\Omega}}{\partial p} \delta p  \, \mathrm{d}V 
+ 
\int_{\Gamma} \dfrac{\partial J_{\Gamma}}{\partial p} \delta p  \, \mathrm{d}S
+\\+
\int_{\Gamma} 
(\mathbf{v}\cdot \mathbf{u} )\mathbf{n} \cdot \delta\mathbf{u}
\, \mathrm{d}S
-
\int_{\Omega} 
(\nabla \mathbf{v}\cdot \mathbf{u})\cdot\delta \mathbf{u}
\, \mathrm{d}V
+
\int_{\Gamma} 
(\mathbf{n} \cdot \mathbf{u}) \mathbf{v}\cdot \delta \mathbf{u} 
\, \mathrm{d}S
-
\int_{\Omega} 
(\mathbf{u} \cdot \nabla)\mathbf{v}\cdot \delta \mathbf{u}
\, \mathrm{d}V
+\\+
\int_{\Gamma} 
2\nu \mathbf{n} \cdot  D(\mathbf{v})\cdot \delta \mathbf{u}
\, \mathrm{d}S
- \int_{\Gamma} 
2\nu \mathbf{v} \cdot  D(\delta \mathbf{u})\cdot \mathbf{n}
\, \mathrm{d}S
-
\int_{\Omega} 
\nabla \cdot \left( 2\nu D(\mathbf{v}) \right) \cdot \delta \mathbf{u}
\, \mathrm{d}V
+\\+
\int_{\Gamma} 
\mathbf{v}\cdot \mathbf{n} \delta p 
\, \mathrm{d}S
-
\int_{\Omega} 
\nabla \cdot \mathbf{v} \delta p
\, \mathrm{d}V
+
\int_{\Gamma} 
- q \delta \mathbf{u \cdot n}
\, \mathrm{d}S
+
\int_{\Omega} 
\nabla q \cdot \delta \mathbf{u}
\, \mathrm{d}V
\end{multline*}
a poté co dáme k sobě členy se stejnou variací a typem integrálu
\begin{align*}
\delta_\phi L = 
&\int_{\Omega} 
\left(
\dfrac{\partial J_{\Omega}}{\partial \mathbf{u}}
- \nabla \mathbf{v}\cdot \mathbf{u}
- (\mathbf{u} \cdot \nabla)\mathbf{v}
- \nabla \cdot \left( 2\nu D(\mathbf{v}) \right)
+ \nabla q
\right)
\cdot \delta \mathbf{u}
\, \mathrm{d}V
\\+
&\int_{\Gamma}
\left(
\dfrac{\partial J_{\Gamma}}{\partial \mathbf{u}}
+ (\mathbf{v}\cdot \mathbf{u} )\mathbf{n} 
+ (\mathbf{n} \cdot \mathbf{u}) \mathbf{v}
+ 2\nu \mathbf{n} \cdot  D(\mathbf{v})
- q \mathbf{n}
\right)
\cdot \delta \mathbf{u}
\, \mathrm{d}S
\\-
&\int_{\Gamma} 
2\nu \mathbf{v} \cdot  D(\delta \mathbf{u})\cdot \mathbf{n}
\, \mathrm{d}S,
\\+
&\int_{\Omega} 
\left(
\frac{\partial J_\Omega}{\partial p}
- \nabla \cdot \mathbf{v}
\right)
\delta p
\, \mathrm{d}V
\\+
&\int_{\Gamma}
\left(
\frac{\partial J_\Gamma}{\partial p}
+ \mathbf{v} \cdot \mathbf{n}
\right)
 \delta p
\, \mathrm{d}S
\end{align*}
dostáváme stejný výraz jako je odvozen v \cite{othmer2008continuous}.

Podmínka $ \delta_\phi L $ musí být splněna pro libovolnou variaci stavových proměnných. Z objemových integrálů nám tedy vychází sdružené NS rovnice pro nestlačitelnou tekutinu ve stacionárním stavu, které můžeme zapsat jako
\begin{align}
2D(\mathbf{v})\mathbf{u}
+ \nabla \cdot \left( 2\nu D(\mathbf{v}) \right)
- \nabla q 
=
\dfrac{\partial J_{\Omega}}{\partial \mathbf{u}} 
\\
\nabla \cdot \mathbf{v} 
= 
\frac{\partial J_\Omega}{\partial p},
\end{align}
kde jsme použili  
\begin{equation*}
\nabla \mathbf{v}\cdot \mathbf{u}
+ (\mathbf{u} \cdot \nabla)\mathbf{v} 
=
\frac{\partial v_j}{\partial x_i}  u_j + u_j  \frac{\partial v_i}{\partial x_j} 
=
2\cdot\frac{1}{2}
\left(
\frac{\partial v_j}{\partial x_i}
+ \frac{\partial v_i}{\partial x_j} 
\right)
u_j
=
2D(\mathbf{v})\mathbf{u}.
\end{equation*}
Hraniční integrály nám pak dávají návod na sestavení okrajových podmínek pro sdružené rovnice a to tak, aby
\begin{align}
\label{eq:sdruzenaOP1}
\int_{\Gamma}
\left(
\dfrac{\partial J_{\Gamma}}{\partial \mathbf{u}}
+ (\mathbf{v}\cdot \mathbf{u} )\mathbf{n} 
+ (\mathbf{n} \cdot \mathbf{u}) \mathbf{v}
+ 2\nu \mathbf{n} \cdot  D(\mathbf{v})
- q \mathbf{n}
\right)
\cdot \delta \mathbf{u}
\, \mathrm{d}S
&= 
\int_{\Gamma} 
2\nu \mathbf{v} \cdot  D(\delta \mathbf{u})\cdot \mathbf{n}
\, \mathrm{d}S
\\
\label{eq:sdruzenaOP2}
\int_{\Gamma}
\left(
\frac{\partial J_\Gamma}{\partial p}
+ \mathbf{v} \cdot \mathbf{n}
\right)
\delta p
\, \mathrm{d}S
&= 0.
\end{align}
Systém sdružených rovnic nápadně připomíná původní systém nestlačitelných NS rovnic. Mezi zásadní rozdíly patří linearita rovnic a také, že v porovnání s původními NS rovnicemi má konvektivní člen opačné znaménko. To značí, že informace se v šíří proti směru stavové rychlosti $ \mathbf{u} $ namísto po směru proudu. Sdružené rovnice často bývají o něco jednodušší, neboť existuje celá řada cenových funkcí, které závisí pouze na hraničních hodnotách. Ku příkladu třecí ztráty, výslednice sil na těleso nebo stlačení. Pro takové cílové funkce pak odpadávají pravé strany sdružených rovnic, neboť $ J_\Omega = 0 $.


\section{Pohyb site}

Nastinit nekolik moznych zpusobu, popsat vic zpusob pomoci obecneho prevodu na B-spliny z OF



\section{Optimalizacni cyklus}

flowchart s naznacenim kde kdy se jake rovnice pocitaji (hlavni NS-rce, sdruzene rce, pohyb site)

\begin{figure}
\includegraphics[width=0.9\textwidth]{./img/flowchart/optimalizacni_cyklus.pdf}
\caption{Optimalizacni cyklus}
\end{figure}









\part{Praktická aplikace}

%!TEX ROOT=../_main.tex

\chapter{Tvarova optimalizace kompresorove mrize}

Optimalizace kompresorové mříže je fajn

Obecny popis pripadu a OP
2D

\section{Obecný popis problému}

Simulace prováděné v rámci této práce zjednodušeně reprezentují měřící soustavu tzv. lopatkových mříží. Pro lopatkové mříže jsou určující zejména geometrie samotné lopatky, rozteč $ t $ jednotlivých lopatek a stav proudu před a za mříží.

V rámci numerické simulace je topologie výpočetní oblasti naznačena na obrázku \ref{fig:vypocetni_oblast}. Vstupní a výstupní hranice jsou rovnoběžné s osou y a jejich výška je právě zmiňovaný parametr rozteče $ t $. Uprostřed oblasti se nachází uzavřená hranice reprezentující geometrii lopatky. Vrchní a spodní hranice jsou brány jako periodické, tedy to co vyteče spodem, vteče vrchem a naopak. Prostorová diskretizace (síť) je v této práci vždy dělána tak, aby si stěny buněk na obou stranách periodické hranice odpovídali $ 1:1 $, pouze s posunutím $ t $.

\begin{figure}
	\includegraphics[width=0.7\textwidth]{img/ComputationalDomain.png}
	\caption{Náčrt topologie výpočetní oblasti pro standardní axiální kompresorovou mříž.}
	\label{fig:vypocetni_oblast}
\end{figure}

Krom periodicity jsou další okrajové podmínky následující. Na vstupu je předepsána Dirichletova okrajová podmínka pro vektor rychlosti a turbulentní proměnné. Tlak zde má nulovou Neumanovu podmínku. Na výstupu je naopak statický tlak fixován na nulu a rychlost je zde předepsána pomocí nulového gradientu.

\section{Cílové funkce}

V rámci aplikace sdružené optimalizace na tvar lopatky kompresorové mříže lze vymyslet hned několik cílů. V jednom stupni axiálního kompresoru se běžně sledují veličiny stlačení, účinnost (respektive ztráty) a výstupní úhel proudu. Pro tuto práci se jako sledovaná a optimalizovaná veličina bere ta poslední, tedy výstupní úhel proudu $ \alpha_2 $. 

K dosažení cílového úhlu na výstupu z lopatkové mříže $ \alpha_{2tar} $ jsou použity dva postupy. Oba vycházejí z definice úhlu výstupního proudu
\begin{equation}
\alpha_2 = \arctan\left(\dfrac{u_{y2}}{u_{x2}}\right).
\end{equation}
Ze zákona zachování hmotnosti pro proudění nestlačitelné tekutiny vyplývá, že to co do kontrolní oblasti vteče, musí i vytéct. Jinými slovy toky vstupní a výstupní hranicí se musejí rovnat
\begin{equation}
\dot{m_1}=\dot{m_2},
\end{equation}
což pro proudění nestlačitelné tekutiny kontrolní oblastí s vstupní a výstupní hranicí rovnoběžnou s osou y znamená, že 
\begin{equation}
u_{x1}=u_{x2}.
\end{equation}
Okrajové podmínky definují na vstupu konstantní uniformní vektor rychlosti $ \mathbf{u_1} $ a přeneseně tedy i x-ovou složku rychlosti na výstupu. Z toho vyplývá, že pro zadané $ \alpha_{2tar} $ můžeme apriori spočítat
\begin{equation}
u_{y2tar} = \tan(\alpha_{2tar}) \cdot u_{x2},
\end{equation}
tedy jistou cílovou rychlost na výstupu a cílovou funkci formulovat pomocí ní.

Dále jsou srovnány dvě formulace cílové funkce. První formulace optimalizuje přímo výstupní složku rychlosti, kdežto druhá optimalizuje nepřímo přes sílu na lopatku.

\subsection{Přímá formulace}

V přímé formulaci se snažíme aby
\begin{equation}
	u_{y2tar}=\dfrac{1}{\phi_2}\sum_{f\in\Gamma_2}\phi_f u_{yf},
\end{equation}
tedy aby průměr $ u_y $ na výstupu vážený přes hmotnostní tok byl roven zadané cílové rychlosti.
Minimalizovanou cílovou funkci formulujeme jako
\begin{equation}
	J = \int_{\Gamma_2}\left( u_y(y)-u_{y2tar} \right)^2\mathrm{d}S = \int_{\Gamma_2} J_\Gamma\, \mathrm{d}S.
\end{equation}
Ve smyslu vztahu \ref{eq:cenova_fce} má takto definovaná cílová funkce pouze hraniční složku $ J_\Gamma $ a pro sdružené rovnice tak bude ;ovat pouze v hraničních členech, tedy v rovnicích \ref{eq:sdruzenaOP1} a \ref{eq:sdruzenaOP2}. Potřebujeme tedy vydefinovat parciální derivace podle primárních proměnných $ \mathbf{u} $ a $ p $. Pro implementaci v rámci knihovny OpenFOAM je pak navíc potřeba vydefinovat ještě derivace podle $ u_n $ a $ u_t $, což pro dříve definovanou úlohu znamená podle složek $ u_x $ a $ u_y $.

Parciální derivace podle primárního tlaku $ \dfrac{\partial}{\partial p} $ je nulová, neboť tlak v cílové funkci nefiguruje.

Pro parciální derivaci podle primární rychlosti $ \mathbf{u} $ je lepší se dívat na složku $ u_y $ jako na skalární součin $ \mathbf{u}\cdot \mathbf{j}=\mathbf{u}\cdot (0,1,0) = u_y$ a derivaci tedy provést jako
\begin{equation}\label{key}
\dfrac{\partial J_\Gamma}{\partial \mathbf{u}}
=
\dfrac{\partial \left( \mathbf{u}\cdot \mathbf{j}-u_{y2tar} \right)^2}{\partial \mathbf{u}}
=
2( \mathbf{u}\cdot \mathbf{j}-u_{y2tar} )\,\mathbf{j}
=
2( u_y-u_{y2tar} )\,\mathbf{j}.
\end{equation}
U dodatečných derivací pro OpenFOAM vychází $ \dfrac{\partial}{\partial u_n}=0 $ a $ \dfrac{\partial }{\partial u_t} $ je stejná jako derivace podle $ \mathbf{u} $.

implementace v OF do apendixu?

\subsection{Nepřímá formulace přes sílu}

Druhou možností jak dosáhnout zadaného úhlu výstupního proudu je použít optimalizaci přes cílovou sílu. 
Optimalizace síly na stěnu ve smyslu její minimalizace v předepsaném směru je v balíku OpenFOAM formulována jako
\begin{equation}\label{key}
J=\dfrac{\int_\Gamma \rho (-\tau_{ij}n_j+pn_i)r_i\,\mathrm{d}S}{\frac{1}{2}\rho A U_{\infty}^2},
\end{equation}
kde $ \tau_{ij} $ jsou složky tenzoru napětí, $ p $ tlak dělený konstantní hustotou $ \rho $ a $ \mathbf{n} $ jednotkový normálový vektor. Vektor $ \mathbf{r} $ pak definuje směr projekce vektoru síly (směr ve kterém se minimalizuje). $ A $ je referenční plocha a $ U_{\infty} $ je rychlost volného proudu. Takto definovaná cílová funkce tedy svou velikostí odpovídá koeficientu síly $ C_f $.¨

Zatímco přímá formulace ovlivňovala přes derivaci cílové funkce okrajovou podmínku sdružených pouze na výstupní hranici, cílová funkce přes sílu ovlivňuje okrajovou podmínku přímo na lopatce. Druhým rozdílem je pak, že v přímé formulaci se objevuje jediná primární proměnná, kdežto pro integraci síly na lopatce jsou potřeba všechny proměnné, včetně turbulentní proměnné $ \widetilde{\nu} $. Tyto rozdíly zavdávají dostatečný důvod pro porovnání těchto dvou formulací ve smyslu rychlosti konvergence či přesnosti.


\section{Optimalizace mrize GHH 1-S1}

popis geometrie, vypocetni oblasti, okrajove podminky, nastaveni optimalizacniho algoritmu

Pro aplikaci optimalizačního algoritmu s novou cenovou funkcí pro stlačení byla zvolena axiální kompresorová mříž MAN GHH 1-S1 publikovaná v \cite{steinert1990design}. Výpočetní oblast 

\subsection{vysledky}
klaibrace S-A modelu, prubeh cilove fce, overeni vysledku pomoci vhodnejsiho modelu turbulence



\chapter{Závěr}
\lipsum[1]
 




%\blindmathtrue
%\blinddocument

%\begin{table}
%\begin{ctucolortab}
%\begin{tabular}{cc}
%\bfseries Foo & \bfseries Bar \\\Midrule
%foo1 & bar1 \\
%foo2 & bar2
%\end{tabular}
%\end{ctucolortab}
%\caption{Foobar.}
%\label{tab:foobar}
%\end{table}

%\begin{figure}
%\includegraphics[width=0.4\textwidth]{ctu_logo_black}
%\caption{Black logo of the CTU in Pragueueue.}
%\end{figure}

%\begin{figure}[!t]
%\includegraphics[width=0.4\textwidth]{ctu_logo_blue}
%\caption{Blue logo of the CTU in Pragueueue.}
%\end{figure}



\appendix

\printnomenclature

\printindex

\bibliographystyle{abbrv}
\bibliography{bibliography/_main}



\end{document}